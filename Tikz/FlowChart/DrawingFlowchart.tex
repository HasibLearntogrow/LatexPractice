\documentclass{article}
\usepackage{tikz}
\usetikzlibrary{shapes.geometric, arrows}
%-----------------------------------------------------------------------FlowChart---------------------------------
% Step-1: craete style for each node
%step -2: create node with lavel number
%step -3: draw line one node to another node

% Solution for step-1: \tikzstyle{nam} = [shape,minimum height,minimum width,text centered,fill=color,draw=colorr]
% solution for step-2: \node(level number)[name]{lavel_name};
% Solution for step-3: \darw[->] (first node) to (second node);

% Step-1: craete style for each node
%step -2: create node with lavel number
%step -3: draw line one node to another node

\title{Drawing FlowChart}
\author{MD Hasib Mia}
\begin{document}
	\maketitle
	\tableofcontents
	\clearpage
	%------------------------------------Step-1:create style for each node----------------------------
	\tikzstyle{rec} = [rectangle,rounded corners,text width=3cm,text centered,minimum height=1cm,draw,fill=pink!20]
    \tikzstyle{elli} = [ellipse, minimum width=3cm, minimum height=1cm, text centered, draw=black, fill=red!30]
    \tikzstyle{Dai} = [diamond, minimum width=2cm, minimum height=.8cm, text centered, draw=black, fill=green!30]
    \tikzstyle{connector} = [circle, minimum size=0.2cm, text centered, draw=black, fill=yellow!30]
    %------------------------------------------------------------------------------------------------------
   
   
   %---------------------step-2: Create node---------------------------------------------
    \section{Create basic node for Flow Chart(step-2):}
    \subsection{Create Start/End section:}
    \begin{tikzpicture}
    	\node (1) [elli] {Start/End};
    \end{tikzpicture}
    
    
    
     \subsection{Create Input/Output section:}
    \begin{tikzpicture}
    	\node (1) [rec] {Input/Output};
    \end{tikzpicture}
    
    
    
    \subsection{Create Condition section:}
    \begin{tikzpicture}
    	\node (1) [Dai] {Condition?};
    \end{tikzpicture}
    
    
    
    \subsection{Create connection section:}
    \begin{tikzpicture}
    	\node (1) [connector] {Connection};
    \end{tikzpicture}
    
    
    %---------------------------------------------------------------------------
    
    
    %------------------------step-3:Linked two node--------------------------
    
    \section{create linked two node:}
    
    \begin{tikzpicture}[node distance=2cm]
    	\node(1)[elli]{Start};
    	\node(2) [rec,below of=1]{Input};
    	\draw[->] (1) to (2);
    	
    	
    \end{tikzpicture}
    
    %--------------------------------------------------------------------------------------------------------------------------------
    
    %---------------------------------------------------practice section-------------------------------------------------------------
    
    \section{Practice of FlowChart:}
     \subsection{Flow chart for adding two two number:}
     \begin{center}
     	\begin{tikzpicture}[node distance=2cm]
     		\node (1) [elli] {Start};
     		
     		\node (2) [rec,below of=1] {Input two value a,b:};
     		\node (3) [rec,below of=2] {sum=a+b};
     		\node (4) [rec,below of=3] {Output: sum};
     		\node (5) [elli,below of=4] {End};
     		\draw[->] (1) to (2);
     		\draw[->] (2) to (3);
     		 \draw[->](3) to (4); 
     		 \draw[->] (4) to(5);
     	\end{tikzpicture}
     \end{center}
     \clearpage
     %---------------------------------------------------------------------2nd practice-------------------------------------------
     
     \subsection{Flow chart with condition:}
     
     \begin{tikzpicture}[node distance=2cm]
     	\node (1) [elli] {Start};
     	
     	\node (2) [rec,below of=1] {Input two value a,b:};
     	\node(3)[Dai,below of=2]{a\textgreater b?};
     	\node (4) [rec,below of=3] {Largest number is b};
     	\node (5) [rec,left of=3,xshift=-3cm] {Largest number is:a};
     	\node(6) [connector,below of=4]{c};
        \node(7)[elli,below of=6]{End};
        \draw[->,thick] (1) to (2);
        \draw[->,thick] (2) to (3);
        \draw[->,thick] (3) to (4);
        \draw[->,thick] (4) to (6);
        \draw[->,thick] (6) to (7);
        \draw[->,thick](3) -- ++(-3.4,0) (5);
        \draw[->,thick] (5) |- (6);
     	
     \end{tikzpicture}
     
	
\end{document}